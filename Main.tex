\documentclass[11pt, oneside]{article}
\usepackage{geometry}
\geometry{letterpaper}
\usepackage{graphicx}
\usepackage{amssymb}
\usepackage{booktabs}
\usepackage{float}
\usepackage[american]{circuitikz}
\usepackage{multirow}

\newcommand{\tabitem}{~~\llap{\textbullet}~~}

\begin{document}

% Definitions
\section{Definitions}
$\omega_c$: Cutoff Frequency\\
$\omega_{c,h}$: High Cutoff Frequency (Bandpass)\\
$\omega_{c,l}$: Low Cutoff Frequency (Bandpass)\\
$\omega_0$: Center Frequency\\
$B_p$: $\omega_{c,h}-\omega_{c,l}$\\
$sbd_{dB}$: Stopband Deviation in Decibles\\
$pbd_{dB}$: Passband Deviation in Decibles\\
$G_0$: Passband Gain\\
$F(s)$: Filter Equation\\
$n$: Filter Order\\
$k_f$: Scaling Factor for the Cuttoff Frequency\\
$\Delta$: Difference / Derivative\\
$x^*$: Complex Conjugate of $x$\\ 

\subsection{Group Delay}
\[\frac{\Delta p}{\Delta f}\] Where $-180 \leq p \leq 180$ degrees and $f$ in Hertz

%\subsection{Ideal Filters}
%\subsubsection{Low-Pass}
%\subsubsection{High-Pass}
%\subsubsection{Bandpass}
%\subsubsection{Notch}
%\subsubsection{Comb}
%\subsubsection{Allpass}

% Filters
\section{Analog Filters}

\begin{table}[h]
\centering
\begin{tabular}{ccc}
        \multirow{4}{*}{Narrower Transition $\downarrow$} & \multicolumn{1}{l}{Bessel} & \multirow{4}{*}{$\uparrow$ More Linear Passband Phase}\\
        & \multicolumn{1}{l}{Butterworth} \\
        & \multicolumn{1}{l}{Chebyshev} \\
        & \multicolumn{1}{l}{Elliptic} \\
\end{tabular}
\end{table}

%\begin{table}[H]
%  \centering
%  \begin{tabular}{lll}
%    \toprule
%    \multicolumn{3}{c}{Filter Comparisons} \\[.5\normalbaselineskip]
%    \bf{Type} & \bf{Pros} & \bf{Cons} \\
%    \midrule
%    1.~Bessel\\
%    \tabitem  & Flat passband phase & Long transition-band\\
%    2.~Butterworth\\
%    \tabitem & Easy to implement & `Average' filter\\
%    \tabitem & Flat passband gain & \\
%    3.~Chebyshev\\
%    \tabitem & Short transition-band & Passband ripple\\
%    4.~Inverse Chebyshev\\
%    \tabitem & Shorter transition-band than Chebyshev & Difficult to implement\\
%    \tabitem & Ripple in stopband instead of passband & Rarely used, Elliptic preferred\\
%    5.~Elliptic\\
%    \tabitem & Shortest transition-band & Ripple in stopband and passband\\
%    \bottomrule
%  \end{tabular}
%\end{table}

\subsection{Bessel}
Choose $n$ then create as follows:
\[ F(s) = \frac{\theta_n(0)}{\theta_n(s)}\]
\[ \theta_n(x) = \sum_{k=0}^{n} \frac{(2n-k)!}{2^{n-k}k!(n-k)!}x^k \]

%Example: \[ \theta_3(\frac{s}{\omega_c=1}) = \sum_{k=0}^3 \frac{(2(3)-k)!}{2^{3-k}k!(3-k)!}s^k = \frac{6!}{15}s^0+\frac{5!}{15}s^1+\frac{4!}{6}s^2+\frac{3!}{3!}s^3 = s^3 + 6s^2 + 15s + 15\]

\subsection{Butterworth}

%% Add in image of pole locations

Choose $n$ and $G_0$ then create as follows:
\[ F(s) = \frac{G_0}{B_n(\frac{s}{w_c})}\]
For even $n$:
\[ B_n(x) = \prod_{k = 1}^{n/2}x^2 - 2cos(\frac{\pi(2k+n-1)}{2n})x+1\]
For odd $n$:
\[ B_n(x) = (x+1)\prod_{k = 1}^{(n-1)/2}x^2 - 2cos(\frac{\pi(2k+n-1)}{2n})x+1\]
%Example: \[B_3(\frac{s}{\omega_c = 2.5}) = (\frac{s}{2.5}+1)\prod_{k = 1}^{(3-1)/2}(\frac{s}{2.5})^2 - 2cos(\frac{\pi(2(1)+(1)-1)}{2(1)})\frac{s}{2.5}+1 = (\frac{s}{2.5}+1)(\frac{s^2}{2.5^2} + 2\frac{s}{2.5} + 1)\]

\subsubsection{Chebyshev}
Choose $G_0$, $n$, $\omega_c$, and $pbd_{dB}$ then create as follows:\\
\[F(s) = \frac{G_0X_n(s)}{V_n(s)}\]
\[\epsilon = \sqrt{10^{\frac{-pbd_{dB}}{10}}-1}\]
\[\alpha = \frac{1}{\epsilon}+\sqrt{1+\epsilon^{-2}}\]
\[A = \frac{\alpha^{\frac{1}{n}} - \alpha^{\frac{-1}{n}}}{2}\]
\[B =\frac{\alpha^{\frac{1}{n}} + \alpha^{\frac{-1}{n}}}{2} \]
For even n:
\[p_k = \omega_c(Acos(\frac{(2k - 1)\pi}{2n})+jBsin(\frac{(2k-1)\pi}{2n}))\]
\[V_n(s) = \prod_{k = 1}^{n/2} s^2 + s(p_k+p_k^*)+p_kp_k^*\]
\[X_n(s) = 10^{-pbd_{dB}/20}\prod_{k = 1}^{n/2}p_kp_k^*\]
For odd n:
\[p_k = \omega_c(Acos(\frac{k\pi}{n})+jBsin(\frac{k\pi}{n}))\]
\[V_n(s) = (s+A\omega_c)\prod_{k = 1}^{(n-1)/2} s^2 + s(p_k+p_k^*)+p_kp_k^*\]
\[X_n(s) = A\omega_c\prod_{k = 1}^{(n-1)/2}p_kp_k^*\]
Side Note:\\
Chebyshev Polynomials:
\[T_n(x) = \left\{
                \begin{array}{ll}
                  T_0(x) = 1\\
                  T_1(x) = x\\
                  T_{n+1}(x) = 2xT_n(x) - T_{n-1}(x)
                \end{array}
              \right. =
               \left\{
                \begin{array}{ll}
                  cos (n cos^{-1}(x)), |x|\leq 1\\
                  cosh(ncosh^{-1}(x)), x\geq 1\\
                  -1^ncosh(ncosh^{-1}(-x)),x\leq -1
                \end{array}
              \right.\]
\[|F(s)| = \sqrt{\frac{1}{1+\epsilon^2T_n^2(\frac{\omega}{\omega_c})}}\]
\subsubsection{Elliptic / Cauer}
\subsubsection{Inverse Chebyshev / Type II Chebyshev}
Choose $G_0$, $n$, $\omega_c$, and $sbd_{dB}$ then create as follows:\\
\[F(s) = \frac{G_0X_n(s)}{V_n(s)}\]
\[A_s = 10^{\frac{sbd_{dB}}{10}}\]
\[\epsilon = \sqrt{\frac{A_s}{1-A_s}}\]
\[u = \frac{sinh^{-1}(\frac{1}{\epsilon})}{n}\]
\[z_k = j\omega_c sec(\frac{(2k-1)\pi}{2n})\]
\[p_k = \frac{\omega_c}{sinh(u)sin(\frac{(2k-1)\pi}{2n})+jcosh(u)cos(\frac{(2k-1)\pi}{2n})}\]
For even n:
\[V_n(s) = \prod_{k=1}^{k=n/2}s^2+s(p_kp_k^*)+p_kp_k^*\]
\[X_n(s) = \prod_{k = 1}^{k = n/2}\frac{p_kp_k^*}{z_kz_k^*}(s^2+s(z_k+z_k^*)+z_kz_k^*)\]
For odd n:
\[V_n(s) = (s+p_n)\prod_{k=1}^{k=(n-1)/2}s^2+s(p_kp_k^*)+p_kp_k^*\]
\[X_n(s) =\sqrt{p_np_n^*} \prod_{k = 1}^{k = (n-1)/2}\frac{p_kp_k^*}{z_kz_k^*}(s^2+s(z_k+z_k^*)+z_kz_k^*)\]

\subsection{Filter Transformations}

\subsubsection{Frequency Scaling}
Replace $s$ with $\frac{s}{k_f}$

\subsubsection{Lowpass to Highpass}
Replace $s$ with $\frac{1}{s}$

\subsubsection{Lowpass to Bandpass}
Replace $s$ with $\frac{s^2+\omega_0^2}{B_ps}$

\subsubsection{Lowpass to Bandstop}
Replace $s$ with $\frac{B_ps}{s^2+\omega_0^2}$

\subsubsection{Lowpass to Notch}
Notch is synonymous with `Bandstop' and `Band-reject'\\
Replace $s$ with $\frac{B_ps}{s^2+\omega_0^2}$

\section{Digital Filters}

%Transformations
\subsection{Bilinear Transformation}
Transforms any analog filter into a IIR digital filter.  Phase response will differ from the analog version.\\
Form desired analog filter then replace $s$ as follows: $s = \frac{2}{T}\frac{1-z^{-1}}{1+z^{-1}}$\\
where T is the sample rate and $z^{-1}$ is the previous data point\\

\subsection{Finite Impulse Response (FIR)}
No feedback and linear phase\\


\subsubsection{Window Method}
$F(n) = [Window]\times [Impulse\ Type]$\\\\
\begin{tabular}{|c|c|c|c|c|}\hline
\bf{Window} & \bf{Peak Sidelobe} & \bf{Main Lobe Width} & \bf{Peak Approximation Error}\\\hline 
Rectangular & 0 & 0 & 0 \\\hline
Hanning & 0 & 0 & 0  \\\hline
Hamming & 0 & 0 & 0  \\\hline
Blackmann & 0 & 0 & 0  \\\hline
\end{tabular}
\\\\\\
\begin{tabular}{|c|c|}\hline
\bf{Window} & \bf{Equation}\\\hline 
Rectangular & 0 \\\hline
Hanning & $sin^2(\frac{n\pi}{N})$ \\\hline
Hamming & $0.54 - 0.46 cos(\frac{2\pi n}{N})$ \\\hline
Blackmann & $0.42 - 0.5 cos(\frac{2\pi n}{N}) + 0.08 cos(\frac{4\pi n}{N})$ \\\hline
\end{tabular}
\\\\\\
\begin{tabular}{|c|c|}\hline
\bf{Impulse Type} & \bf{Equation}\\\hline 
Low Pass & $\frac{\omega_c}{\pi}sinc(\frac{n\omega_c}{\pi})$ \\\hline
High Pass & $sinc(n) - \frac{\omega_c}{\pi}sinc(\frac{n\omega_c}{\pi})$ \\\hline
Band Pass & $\frac{\omega_{c,h}}{\pi}sinc(\frac{n\omega_{c,h}}{\pi}) - \frac{\omega_{c,l}}{\pi}sinc(\frac{n\omega_{c,l}}{\pi})$ \\\hline
\end{tabular}
% $h[k] = \frac{sin(k\omega_c)}{k\pi}$

\section{Implementations}
\subsection{Passive}
\subsubsection{Pi Filter}
\subsubsection{T FIlter}
\subsection{Active}
\subsubsection{Sallen-Key}
$\frac{V_{OUT}}{V_{IN}} = \frac{Z_3Z_4}{Z_1Z_2+Z_3(Z_1+Z_2)+Z_3Z_4}$\\
For Lowpass: $Z_1, Z_2$ become resistors and $Z_3, Z_4$ become capacitors\\
For Highpass: $Z_3, Z_4$ become resistors and $Z_1, Z_2$ become capacitors\\
\begin{figure}[h!]
  \begin{center}
	\begin{circuitikz}
	% operational amplifier
	\node[op amp, yscale=-1] (opamp) at (6,-0.5) {};
	% input port to positive op amp input
	\draw ($(opamp.+)+(-4,0)$) node[left] {$V_{in}$} to[european resistor, l=$Z_1$, o-*] ($(opamp.+)+(-2.5,0)$) node (betweenc) {}
	to[european resistor, l=$Z_2$, *-*] ($(opamp.+)+(-1,0)$) node (behindc) {}
	to[short] (opamp.+);
	% resistor from positive op amp input to ground
	\draw (behindc) to[european resistor, l_=$Z_4$, *-] ($(behindc)+(0,-2)$) node [ground] {};
	% positive feedback resistor from the op amp output to the node between the capacitors
	\draw (betweenc) to[short] ($(betweenc)+(0,1)$) to[european resistor, l=$Z_3$] ($(opamp.out)+(0,1.5)$) to[short] (opamp.out);
	% negative feedback loop from the op amp output to the negative op amp input
	\draw (opamp.-) to[short] ($(opamp.-)+(0,-1)$) to[short] ($(opamp.out)+(0,-1.5)$) to[short] (opamp.out);
	% output port
	\draw (opamp.out) to[short, *-o]  ($(opamp.out)+(0.5,0)$) node[right] {$V_{out}$};
	\end{circuitikz}
    \caption{Sallen-Key Generic}
  \end{center}
\end{figure}

\subsubsection{Phase Lead-Lag Compensator}


\end{document}